\documentclass{article}

%\usepackage{amsfonts}
\usepackage{amsmath}

\usepackage[dvipsnames]{xcolor}
\numberwithin{equation}{section} % to change the numbering of the section from x to x.y

\begin{document}

\section{What is deep learning?}

\subsubsection{Artificial intelligence}

A concise definition of the field would be as follows: \textit{the effort to automate intellectual tasks normally performed by humans}. For a fairly long time, many experts believed that human-level artificial intelligence could be achieved by having programmers handcraft a sufficiently large set of explicit rules for manipulating knowledge. This approach is known as symbolic AI, and it was the dominant paradigm in AI from the 1950s to the late 1980s. It reached its peak popularity during the expert systems boom of the 1980s. Although symbolic AI proved suitable to solve well-defined, logical problems, such as playing chess, it turned out to be intractable to figure out explicit rules for solving more complex, fuzzy problems, such as image classification, speech recognition, and language translation. A new approach arose to take symbolic AI ’s place: \textit{machine learning}. \\

\subsubsection{Machine learning}

With machine learning, humans input data as well as the answers expected from the data, and out come the rules. These rules can then be applied to new data to produce original answers. \\

Although machine learning only started to flourish in the 1990s, it has quickly
become the most popular and most successful subfield of AI, a trend driven by the
availability of faster hardware and larger datasets. Machine learning is tightly related to mathematical statistics, but it differs from statistics in several important ways. Unlike statistics, machine learning tends to deal with large, complex datasets (such as a dataset of millions of images, each consisting of tens of thousands of pixels) for which classical statistical analysis such as Bayesian analysis would be impractical. As a result, machine learning, and especially deep learning, exhibits comparatively little mathematical theory—maybe too little—and is engineering oriented. It’s a hands-on discipline in which ideas are proven empirically more often than theoretically. \\

\subsubsection{Learning representations from data}

To do machine learning, we need three things: \\

\begin{itemize}
	\item Input data points
	\item Examples of the expected output
	\item A way to measure whether the algorithm is doing a good job:  This adjustment step is what we call \textit{learning}.
\end{itemize}

Therefore, the central problem in machine learning and deep learning is to \textit{meaningfully transform data}. Machine-learning models are all about finding appropriate representations for their input data. \textit{Learning}, in the context of machine learning, describes an automatic search process for better representations. Machine-learning algorithms aren’t usually creative in finding these transformations; they’re merely searching through a predefined set of
operations, called a \textit{hypothesis space}. So that’s what machine learning is, technically: searching for useful representations of some input data, within a predefined space of possibilities, using guidance from a feedback signal. This simple idea allows for solving a remarkably broad range of intellectual tasks, from speech recognition to autonomous car driving. Now that you understand what we mean by \textit{learning}, let’s take a look at what makes \textit{deep learning} special. \\


\subsubsection{The “deep” in deep learning}

Deep learning is a specific subfield of machine learning: a new take on learning representations from data that puts an emphasis on learning successive \textit{layers} of increasingly meaningful representations. The \textit{deep} in \textit{deep learning} isn’t a reference to any kind of deeper understanding achieved by the approach; rather, it stands for this idea of successive layers of representations. How many layers contribute to a model of the data is called the \textit{depth} of the model. \\

Meanwhile, other approaches to machine learning tend to focus on learning only one or two layers of representations of the data; hence, they’re sometimes called \textit{shallow learning}. \\

In deep learning, these layered representations are (almost always) learned via
models called \textit{neural networks}, structured in literal layers stacked on top of each other. The term \textit{neural network} is a reference to neurobiology, but although some of the central concepts in deep learning were developed in part by drawing inspiration from our understanding of the brain, deep-learning models are \textit{not} models of the brain. \\

You can think of a deep network as a multistage information-distillation operation, where information goes through successive filters and comes out increasingly \textit{purified} (that is, useful with regard to some task). \\

So that’s what deep learning is, technically: a multistage way to learn data representations. It’s a simple idea—but, as it turns out, very simple mechanisms, sufficiently scaled, can end up looking like magic. \\

\subsubsection{Understanding how deep learning works, in three figures}

The specification of what a layer does to its input data is stored in the layer’s
\textit{weights}, which in essence are a bunch of numbers. In technical terms, we’d say that the transformation implemented by a layer is \textit{parameterized} by its weights (see figure 1.7). (Weights are also sometimes called the \textit{parameters} of a layer.) In this context, \textit{learning} means finding a set of values for the weights of all layers in a network, such that the network will correctly map example inputs to their associated targets. \\

To control the output of a neural network, you need to be able to measure how far this output is from what you expected. This is the job of the \textit{loss function} of the network, also called the \textit{objective function}. The loss function takes the predictions of the network and the true target (what you wanted the network to output) and computes a distance score, capturing how well the network has done on this specific example (see figure 1.8). \\

The fundamental trick in deep learning is to use this score as a feedback signal to
adjust the value of the weights a little, in a direction that will lower the loss score for the current example (see figure 1.9). This adjustment is the job of the \textit{optimizer}, which implements what’s called the \textit{Backpropagation} algorithm: the central algorithm in deep learning. \\

\subsection{Before deep learning: a brief history of machine learning}

\subsubsection{Probabilistic modeling}

Naive Bayes is a type of machine-learning classifier based on applying Bayes’ theo-
rem while assuming that the features in the input data are all independent (a strong, or “naive” assumption, which is where the name comes from). \\

\subsubsection{Kernel methods}

The technique of mapping data to a high-dimensional representation where a classifi-
cation problem becomes simpler may look good on paper, but in practice it’s often computationally intractable. That’s where the kernel trick comes in (the key idea
that kernel methods are named after). Here’s the gist of it: to find good decision
hyperplanes in the new representation space, you don’t have to explicitly compute
the coordinates of your points in the new space; you just need to compute the distance between pairs of points in that space, which can be done efficiently using a \textit{kernel function}. A kernel function is a computationally tractable operation that maps any two points in your initial space to the distance between these points in your target representation space, completely bypassing the explicit computation of the new representation. \\

Because an SVM is a shallow method, applying an SVM to perceptual problems requires first extracting useful representations manually (a step called \textit{feature engineering}), which is difficult and brittle. \\

\subsubsection{Decision trees, random forests, and gradient boosting machines}

Decisions trees learned from data began to receive significant research interest
in the 2000s, and by 2010 they were often preferred to kernel methods. In particular, the Random Forest algorithm introduced a robust, practical take on
decision-tree learning that involves building a large number of specialized decision
trees and then ensembling their outputs. Random forests are applicable to a wide
range of problems—you could say that they’re almost always the second-best algorithm
for any shallow machine-learning task. When the popular machine-learning competition website Kaggle (http://kaggle.com) got started in 2010, random forests quickly became a favorite on the platform—until 2014, when \textit{gradient boosting machines} took over. A gradient boosting machine, much like a random forest, is a machine-learning technique based on ensembling weak prediction models, generally decision trees. It uses \textit{gradient boosting}, a way to improve any machine-learning model by iteratively training new models that specialize in addressing the weak points of the previous models. Applied to decision trees, the use of the gradient boosting technique results in models that strictly outperform random forests most of the time, while having similar properties. It may be one of the best, if not \textit{the} best, algorithm for dealing with nonperceptual data today. Alongside deep learning, it’s one of the most commonly used techniques in Kaggle competitions. \\

\subsubsection{Back to neural networks}

Since 2012, deep convolutional neural networks (\textit{convnets}) have become the go-to algorithm for all computer vision tasks; more generally, they work on all perceptual tasks.  It has completely replaced SVMs and decision trees in a
wide range of applications. For instance, for several years, the European Organization for Nuclear Research, CERN, used decision tree–based methods for analysis of particle data from the ATLAS detector at the Large Hadron Collider (LHC); but CERN eventually switched to Keras-based deep neural networks due to their higher performance and ease of training on large datasets. \\

\subsubsection{What makes deep learning different}

The primary reason deep learning took off so quickly is that it offered better performance on many problems. But that’s not the only reason. Deep learning also makes problem-solving much easier, because it completely automates what used to be the most crucial step in a machine-learning workflow: feature engineering. \\

Humans had to manually engineer good layers of representations for their data. This is called feature engineering. if the crux of the issue is to have multiple successive layers of representations, could shallow methods be applied repeatedly to emulate the effects of deep learning? In practice, there are fast-diminishing returns to successive applications of shallow-learning methods, because \textit{the optimal first representation layer in a three layer model isn’t the optimal first layer in a one-layer or two-layer model}. What is transformative about deep learning is that it allows a model to learn all layers of representation
jointly, at the same time, rather than in succession (greedily, as it’s called). With joint feature learning, whenever the model adjusts one of its internal features, all other features that depend on it automatically adapt to the change, without requiring human intervention. Everything is supervised by a single feedback signal: every change in the model serves the end goal. \\

These are the two essential characteristics of how deep learning learns from data:
the \textit{incremental, layer-by-layer way in which increasingly complex representations are developed}, and the fact that \textit{these intermediate incremental representations are learned jointly}, each layer being updated to follow both the representational needs of the layer above and the needs of the layer below. Together, these two properties have made deep learning vastly more successful than previous approaches to machine learning. \\

\subsubsection{The modern machine-learning landscape}

In 2016 and 2017, Kaggle was dominated by two approaches: gradient boosting
machines and deep learning. Specifically, gradient boosting is used for problems
where structured data is available, whereas deep learning is used for perceptual problems such as image classification. Practitioners of the former almost always use the excellent XGBoost library, which offers support for the two most popular languages of data science: Python and R. Meanwhile, most of the Kaggle entrants using deep learning use the Keras library, due to its ease of use, flexibility, and support of Python. These are the two techniques you should be the most familiar with in order to be successful in applied machine learning today: gradient boosting machines, for shallow-learning problems; and deep learning, for perceptual problems. In technical terms, this means you’ll need to be familiar with XGBoost and Keras—the two libraries that currently dominate Kaggle competitions. With this book in hand, you’re already one big step closer.


\newpage

\section{Before we begin: the mathematical building blocks of neural networks}

\subsection{A first look at a neural network}

We’ll use the MNIST dataset, a classic in the machine-learning community, which has been around almost as long as the field itself and has been intensively studied. It’s a set of 60,000 training images, plus 10,000 test images, assembled by the National Institute of Standards and Technology (the NIST in MNIST) in the 1980s. \\

In machine learning, a category in a classification problem is called a class. Data
points are called samples. The class associated with a specific sample is called a
label. \\

\subsection{Data representations for neural networks}

Tensors are a generalization of matrices to an arbitrary number of dimensions
(note that in the context of tensors, a dimension is often called an axis). \\

\subsubsection{Scalars (0D tensors)}

A tensor that contains only one number is called a scalar (or scalar tensor, or 0-dimensional tensor, or 0D tensor). In Numpy, a float32 or float64 number is a scalar tensor (or scalar array). You can display the number of axes of a Numpy tensor via the ndim attribute; a scalar tensor has 0 axes (ndim == 0 ). The number of axes of a tensor is also called its rank. \\

\subsubsection{Vectors (1D tensors)}

An array of numbers is called a vector, or 1D tensor. A 1D tensor is said to have exactly one axis. Don’t confuse a 5D vector with a 5D tensor! A 5D vector has only one axis and has five dimensions along its axis, whereas a 5D tensor has five axes (and may have any number of dimensions along each axis). Dimensionality can denote either the number of entries along a specific axis (as in the case of our 5D vector) or the number of axes in a tensor (such as a 5D tensor), which can be confusing at times. In the latter case, it’s technically more correct to talk about a tensor of rank 5 (the rank of a tensor being the number of axes), but the ambiguous notation 5D tensor is common regardless. \\

\subsubsection{Matrices (2D tensors)}

An array of vectors is a matrix, or 2D tensor. A matrix has two axes (often referred to rows and columns). You can visually interpret a matrix as a rectangular grid of numbers. The entries from the first axis are called the rows, and the entries from the second axis are called the columns. \\

\subsubsection{3D tensors and higher-dimensional tensors}

If you pack such matrices in a new array, you obtain a 3D tensor, which you can visually interpret as a cube of numbers. By packing 3D tensors in an array, you can create a 4D tensor, and so on. In deep learning, you’ll generally manipulate tensors that are 0D to 4D, although you may go up to 5D if you process video data. \\

\subsubsection{The notion of data batches}

When considering such a batch tensor, the first axis (axis 0) is called the \textbf{batch axis} or \textbf{batch dimension}.

\subsubsection{Real-world examples of data tensors}

The data you’ll manipulate will almost always fall into one of the following categories: \\

\begin{itemize}
	\item \textit{Vector data}--\textbf{2D tensors of shape} (samples, features)
	\item \textit{Timeseries data or sequence data}--\textbf{3D tensors of shape} (samples, timesteps, features)
	\item \textit{Images}--\textbf{4D tensors of shape} (samples, height, width, channels) or (samples, channels, height, width)
	\item \textit{Video}--\textbf{—5D tensors of shape} (samples, frames, height, width, channels) or (samples, frames, channels, height, width)
\end{itemize}

\subsubsection{Vector data}

This is the most common case. In such a dataset, each single data point can be encoded as a vector, and thus a batch of data will be encoded as a 2D tensor (that is, an array of vectors), where the first axis is the \textit{samples axis} and the second axis is the \textit{features axis}. 

\subsubsection{Timeseries data or sequence data}

The time axis is always the second axis (axis of index 1), by convention.


\subsection{The gears of neural networks: tensor operations}

\subsubsection{Tensor reshaping}

A special case of reshaping that’s commonly encountered is \textit{transposition}. \textit{Transposing} a matrix means exchanging its rows and its columns, so that x[i, :] becomes x[:, i] \\

\subsection{The engine of neural networks: gradient-based optimization}

In this expression, W and b are tensors that are attributes of the layer. They’re called the \textit{weights} or \textit{trainable parameters} of the layer (the kernel and bias attributes, respectively). These weights contain the information learned by the network from exposure to training data. \\

Initially, these weight matrices are filled with small random values (a step called random initialization). The resulting representations are meaningless—but they’re a starting point. What comes next is to gradually adjust these weights, based on a feedback signal. This gradual adjustment, also called \textit{training}, is basically the learning that machine learning is all about. \\

This happens within what’s called a \textit{training loop}, which works as follows. Repeat these steps in a loop, as long as necessary:

\begin{enumerate}
	\item Draw a batch of training samples x and corresponding targets y.
	\item Run the network on x (a step called the forward pass) to obtain predictions y\_pred .
	\item Compute the loss of the network on the batch, a measure of the mismatch
	between y\_pred and y .
	\item Update all weights of the network in a way that slightly reduces the loss on this batch.
\end{enumerate}

You’ll eventually end up with a network that has a very low loss on its training data: a low mismatch between predictions y\_pred and expected targets y. Given an individual weight coefficient in the network, how can you compute whether the coefficient should be increased or decreased, and by how much? \\

A much better approach is to take advantage of the fact that all operations used in the network are \textit{differentiable}, and compute the \textit{gradient} of the loss with regard to the network’s coefficients. \\


\subsubsection{Derivative of a tensor operation: the gradient}

A gradient is the derivative of a tensor operation. It’s the generalization of the concept of derivatives to functions of multidimensional inputs: that is, to functions that take tensors as inputs. \\

That tensor gradient(f)(W0) is the gradient of the function f(W) = loss\_value in W0. You saw earlier that the derivative of a function f(x) of a single coefficient can be interpreted as the slope of the curve of f. Likewise, gradient(f)(W0) can be interpreted as the tensor describing the \textit{curvature} of f(W) around W0. \\

\subsubsection{Stochastic gradient descent}

it’s known that a function’s minimum is a point where the derivative is 0, so all
you have to do is find all the points where the derivative goes to 0 and check for which of these points the function has the lowest value. Applied to a neural network, that means finding analytically the combination of weight values that yields the smallest possible loss function. This can be done by solving the equation gradient(f)(W) = 0 for W.  If you update the weights in the opposite direction from the gradient, the loss will be a little less every time:

\begin{enumerate}
	\item Draw a batch of training samples x and corresponding targets y.
	\item Run the network on x to obtain predictions y\_pred.
	\item  Compute the loss of the network on the batch, a measure of the mismatch
	between y\_pred and y .
	\item Compute the gradient of the loss with regard to the network’s parameters (a \textit{backward pass}).
	\item Move the parameters a little in the opposite direction from the gradient—for example W -= step * gradient—thus reducing the loss on the batch a bit.
\end{enumerate}

What I just described is called \textit{mini-batch stochastic gradient descent} (mini-batch SGD). The term \textit{stochastic} refers to the fact that each batch of data is drawn at random (\textit{stochastic} is a scientific synonym of \textit{random}). \\


\subsubsection{Chaining derivatives: the Backpropagation algorithm}

Applying the chain rule to the computation of the gradient values of a neural network gives rise to an algorithm called \textit{Backpropagation} (also sometimes called \textit{reverse-mode differentiation}). Backpropagation starts with the final loss value and works backward from the top layers to the bottom layers, applying the chain rule to compute the contribution that each parameter had in the loss value.



\newpage

\section{Getting started with neural networks}

\subsection{Anatomy of a neural network}

Training a neural network revolves around the following objects:

\begin{itemize}
	\item \textit{Layers}, which are combined into a \textit{network} (or \textit{model})
	\item The \textit{input data} and corresponding \textit{targets}
	\item The \textit{loss function}, which defines the feedback signal used for learning
	\item The \textit{optimizer}, which determines how learning proceeds
\end{itemize}


\subsubsection{Layers: the building blocks of deep learning}

Simple vector data, stored in 2D tensors of shape (samples, features), is often processed by densely connected layers, also called fully connected or dense layers (the Dense class in Keras). Sequence data, stored in 3D tensors of shape (samples,
timesteps, features), is typically processed by recurrent layers such as an LSTM layer. Image data, stored in 4D tensors, is usually processed by 2D convolution layers (Conv2D). \\

The notion of \textit{layer compatibility} here refers specifically to the fact that every layer will only accept input tensors of a certain shape and will return output tensors of a certain shape. \\

\subsubsection{Models: networks of layers}

Some common network topologies include the following:

\begin{itemize}
	\item Two-branch networks
	\item Multihead networks
	\item Inception blocks
\end{itemize}

The topology of a network defines a \textit{hypothesis space}. Picking the right network architecture is more an art than a science; and although there are some best practices and principles you can rely on, only practice can help you become a proper neural-network architect. \\


\subsubsection{Loss functions and optimizers: keys to configuring the learning process}

A neural network that has multiple outputs may have multiple loss functions (one per
output). But the gradient-descent process must be based on a \textit{single} scalar loss value; so, for multiloss networks, all losses are combined (via averaging) into a single scalar quantity. Just remember that all neural networks you build will be just as ruthless in lowering their loss function—so choose the objective wisely, or you’ll have to face unintended side effects. \\

Fortunately, when it comes to common problems such as classification, regression,
and sequence prediction, there are simple guidelines you can follow to choose the
correct loss. For instance, you’ll use binary crossentropy for a two-class classification problem, categorical crossentropy for a many-class classification problem, mean-squared error for a regression problem, connectionist temporal classification (CTC) for a sequence-learning problem, and so on. Only when you’re working on truly new research problems will you have to develop your own objective functions. \\

\subsection{Introduction to Keras}

\subsubsection{Developing with Keras: a quick overview}

There are two ways to define a model: using the Sequential class (only for linear
stacks of layers, which is the most common network architecture by far) or the func-
tional API (for directed acyclic graphs of layers, which lets you build completely arbitrary architectures).


\subsection{Setting up a deep-learning workstation}




\subsection{Predicting house prices: a regression example}

The dataset you’ll use has an interesting difference from the two previous examples. It has relatively few data points: only 506, split between 404 training samples and 102 test samples. And each \textit{feature} in the input data (for example, the crime rate) has a different scale. For instance, some values are proportions, which take values between 0 and 1; others take values between 1 and 12, others between 0 and 100, and so on.

\subsubsection{Preparing the data}

A widespread best practice to deal with such data is to do feature-wise normalization: for each feature in the input data (a column in the input data matrix), you subtract the mean of the feature and divide by the standard deviation, so that the feature is centered around 0 and has a unit standard deviation. This is easily done in Numpy.

\subsubsection{Building your network}

The network ends with a single unit and no activation (it will be a linear layer). This is a typical setup for scalar regression (a regression where you’re trying to predict a single continuous value). Applying an activation function would constrain the range the output can take; for instance, if you applied a sigmoid activation function to the last layer, the network could only learn to predict values between 0 and 1. Here, because the last layer is purely linear, the network is free to learn to predict values in any range. \\

Note that you compile the network with the mse loss function—mean squared error, the square of the difference between the predictions and the targets. This is a widely used loss function for regression problems. You’re also monitoring a new metric during training: mean absolute error (MAE). It’s the absolute value of the difference between the predictions and the targets. For instance, an MAE of 0.5 on this problem would mean your predictions are off by \$500 on average. \\

\subsection{Validating your approach using K-fold validation}

Because you have so few data points, the validation set would end up being very small (for instance, about 100 examples). As a consequence, the validation scores might change a lot depending on which data points you chose to use for validation and which you chose for training: the validation scores might have a high \textit{variance} with regard to the validation split. This would prevent you from reliably evaluating your model. \\

The best practice in such situations is to use K-\textit{fold} cross-validation (see figure 3.11). It consists of splitting the available data into K partitions (typically K = 4 or 5), instantiating K identical models, and training each one on K – 1 partitions while evaluating on the remaining partition. The validation score for the model used is then the average of the K validation scores obtained. \\

Once you’re finished tuning other parameters of the model (in addition to the
number of epochs, you could also adjust the size of the hidden layers), [page 91] \\


\newpage

\section{Fundamentals of machine learning}

\subsection{Evaluating machine-learning models}

In machine learning, the goal is to achieve models that generalize—that perform
well on never-before-seen data—and overfitting is the central obstacle. You can only
control that which you can observe, so it’s crucial to be able to reliably measure the generalization power of your model.

\subsubsection{Training, validation, and test sets}

You may ask, why not have two sets: a training set and a test set? You’d train on the training data and evaluate on the test data. Much simpler! \\

The reason is that developing a model always involves tuning its configuration: for
example, choosing the number of layers or the size of the layers (called the \textit{hyperparameters} of the model, to distinguish them from the \textit{parameters}, which are the network’s weights). You do this tuning by using as a feedback signal the performance of the model on the validation data. In essence, this tuning is a form of \textit{learning}: a search for a good configuration in some parameter space. As a result, tuning the configuration of the model based on its performance on the validation set can quickly result in \textit{overfitting to the validation set}, even though your model is never directly trained on it. \\

Central to this phenomenon is the notion of \textit{information leaks}. Every time you tune a hyperparameter of your model based on the model’s performance on the validation set, some information about the validation data leaks into the model. If you do this only once, for one parameter, then very few bits of information will leak, and your validation set will remain reliable to evaluate the model. But if you repeat this many times—running one experiment, evaluating on the validation set, and modifying your model as a result—then you’ll leak an increasingly significant amount of information about the validation set into the model. \\

At the end of the day, you’ll end up with a model that performs artificially well on
the validation data, because that’s what you optimized it for. You care about performance on completely new data, not the validation data, so you need to use a completely different, never-before-seen dataset to evaluate the model: the test dataset. Your model shouldn’t have had access to \textit{any} information about the test set, even indirectly. \\

If anything about the model has been tuned based on test set performance, then your
measure of generalization will be flawed. \\

Splitting your data into training, validation, and test sets may seem straightforward, but there are a few advanced ways to do it that can come in handy when little data is available. Let’s review three classic evaluation recipes: simple hold-out validation, K-fold validation, and iterated K-fold validation with shuffling. 

\subsubsection*{SIMPLE HOLD-OUT VALIDATION}

Set apart some fraction of your data as your test set. Train on the remaining data, and evaluate on the test set. As you saw in the previous sections, in order to prevent information leaks, you shouldn’t tune your model based on the test set, and therefore you should \textit{also} reserve a validation set. Schematically, hold-out validation looks like figure 4.1.


\subsubsection*{K-FOLD VALIDATION}

With this approach, you split your data into K partitions of equal size. For each partition i, train a model on the remaining K – 1 partitions, and evaluate it on partition i. Your final score is then the averages of the K scores obtained. This method is helpful when the performance of your model shows significant variance based on your train-test split. Schematically, K-fold cross-validation looks like figure 4.2. \\

\subsubsection*{ITERATED K-FOLD VALIDATION WITH SHUFFLING}

This one is for situations in which you have relatively little data available and you need to evaluate your model as precisely as possible. I’ve found it to be extremely helpful in Kaggle competitions. It consists of applying K-fold validation multiple times, shuffling the data every time before splitting it K ways. The final score is the average of the scores obtained at each run of K-fold validation. Note that you end up training and evaluating P $\times$ K models (where P is the number of iterations you use), which can very expensive.


\subsection{Data preprocessing, feature engineering, and feature learning}

\subsubsection{Data preprocessing for neural networks}

\subsubsection*{VECTORIZATION}

All inputs and targets in a neural network must be tensors of floating-point data (or, in specific cases, tensors of integers). Whatever data you need to process—sound, images, text—you must first turn into tensors, a step called \textit{data vectorization}.

\subsubsection*{VALUE NORMALIZATION}

In general, it isn’t safe to feed into a neural network data that takes relatively large values (for example, multidigit integers, which are much larger than the initial values taken by the weights of a network) or data that is heterogeneous (for example, data where one feature is in the range 0–1 and another is in the range 100–200). Doing so can trigger large gradient updates that will prevent the network from converging. To make learning easier for your network, your data should have the following characteristics:

\begin{itemize}
	\item Take \textit{small values}—Typically, most values should be in the 0–1 range.
	\item Be \textit{homogenous}—That is, all features should take values in roughly the same
	range.
\end{itemize}

Additionally, the following stricter normalization practice is common and can help,
although it isn’t always necessary (for example, you didn’t do this in the digit-classification example):

\begin{itemize}
	\item Normalize each feature independently to have a mean of 0.
	\item Normalize each feature independently to have a standard deviation of 1.
\end{itemize}

This is easy to do with Numpy arrays: \\

x -= x.mean(axis=0) \\
x /= x.std(axis=0) \\

\subsubsection*{HANDLING MISSING VALUES}

In general, with neural networks, it’s safe to input missing values as 0, with the condition that 0 isn’t already a meaningful value. The network will learn from exposure to the data that the value 0 means \textit{missing data} and will start ignoring the value. \\

Note that if you’re expecting missing values in the test data, but the network was
trained on data without any missing values, the network won’t have learned to ignore
missing values! In this situation, you should artificially generate training samples with missing entries: copy some training samples several times, and drop some of the features that you expect are likely to be missing in the test data. \\

\subsubsection{Feature engineering}

That’s the essence of feature engineering: making a problem easier by expressing
it in a simpler way. It usually requires understanding the problem in depth. \\

Does this mean you don’t have to worry about feature engineering as
long as you’re using deep neural networks? No, for two reasons:

\begin{itemize}
	\item Good features still allow you to solve problems more elegantly while using fewer resources. For instance, it would be ridiculous to solve the problem of reading a clock face using a convolutional neural network.
	\item Good features let you solve a problem with far less data. The ability of deep-learning models to learn features on their own relies on having lots of training data available; if you have only a few samples, then the information value in their features becomes critical.
\end{itemize}


\subsection{ Overfitting and underfitting}

The fundamental issue in machine learning is the tension between optimization and generalization. Optimization refers to the process of adjusting a model to get the
best performance possible on the training data (the learning in machine learning),
whereas generalization refers to how well the trained model performs on data it has
never seen before. The goal of the game is to get good generalization, of course, but you don’t control generalization; you can only adjust the model based on its training data. \\

To prevent a model from learning misleading or irrelevant patterns found in the
training data, \textit{the best solution is to get more training data}. A model trained on more data will naturally generalize better. When that isn’t possible, the next-best solution is to modulate the quantity of information that your model is allowed to store or to add constraints on what information it’s allowed to store. \\

If a network can only afford to memorize a small number of patterns, the optimization process will force it to focus on the most prominent patterns, which have a better chance of generalizing well. The processing of fighting overfitting this way is called \textit{regularization}. \\

\subsubsection{Reducing the network’s size}

The simplest way to prevent overfitting is to reduce the size of the model: the number of learnable parameters in the model (which is determined by the number of layers and the number of units per layer). In deep learning, the number of learnable parameters in a model is often referred to as the model’s \textit{capacity}. Intuitively, a model with more parameters has more \textit{memorization capacity} and therefore can easily learn a perfect dictionary-like mapping between training samples and their targets—a mapping without any generalization power. \\

Unfortunately, there is no magical formula to determine the right number of layers or the right size for each layer. You must evaluate an array of different architectures (on your validation set, not on your test set, of course) in order to find the correct model size for your data. The general workflow to find an appropriate model size is to start with relatively few layers and parameters, and increase the size of the layers or add new layers until you see diminishing returns with regard to validation loss. \\

As you can see, the bigger network gets its training loss near zero very quickly. The more capacity the network has, the more quickly it can model the training data (resulting in a low training loss), but the more susceptible it is to overfitting (resulting in a large difference between the training and validation loss). \\

\subsubsection{Adding weight regularization}

You may be familiar with the principle of \textit{Occam’s razor}: given two explanations for something, the explanation most likely to be correct is the simplest one—the one that makes fewer assumptions. This idea also applies to the models learned by neural networks: given some training data and a network architecture, multiple sets of weight values (multiple models) could explain the data. Simpler models are less likely to over-fit than complex ones. \\

A \textit{simple model} in this context is a model where the distribution of parameter values has less entropy (or a model with fewer parameters, as you saw in the previous section). Thus a common way to mitigate overfitting is to put constraints on the complexity of a network by forcing its weights to take only small values, which makes the distribution of weight values more \textit{regular}. This is called \textit{weight regularization}, and it’s done by adding to the loss function of the network a \textit{cost} associated with having large weights. This cost comes in two flavors:

\begin{itemize}
	\item L1 \textit{regularization}--The cost added is proportional to the \textit{absolute value of the weight coefficients} (the L1 \textit{norm} of the weights).
	\item L2 \textit{regularization}--The cost added is proportional to the \textit{square of the value of the weight coefficients} (the L2 \textit{norm} of the weights). L2 regularization is also called \textit{weight decay} in the context of neural networks.
\end{itemize}

In Keras, weight regularization is added by passing weight regularizer instances to layers as keyword arguments. l2(0.001) means every coefficient in the weight matrix of the layer will add 0.001 $\ast$ weight\_coefficient\_value to the total loss of the network. Note that because this penalty is \textit{only added at training time}, the loss for this network will be much higher at training than at test time. \\

\subsubsection{Adding dropout}

Dropout is one of the most effective and most commonly used regularization techniques for neural networks, developed by Geoff Hinton and his students at the University of Toronto. Dropout, applied to a layer, consists of randomly dropping out (setting to zero) a number of output features of the layer during training. \\

The dropout rate is the fraction of the features that are zeroed out; it’s usually set between 0.2 and 0.5. At test time, no units are dropped out; instead, the layer’s output values are scaled down by a factor equal to the dropout rate, to balance for the fact that more units are active than at training time. \\

In Keras, you can introduce dropout in a network via the Dropout layer, which is
applied to the output of the layer right before it: [see page 110] \\

To recap, these are the most common ways to prevent overfitting in neural networks:

\begin{itemize}
	\item Get more training data.
	\item Reduce the capacity of the network.
	\item Add weight regularization.
	\item Add dropout.
\end{itemize}

\subsection{The universal workflow of machine learning}

\subsubsection{Defining the problem and assembling a dataset}

For instance, if you’re trying to predict the movements of a stock on the stock market given its recent price history, you’re unlikely to succeed, because price history doesn’t contain much predictive information. \\

One class of unsolvable problems you should be aware of is nonstationary problems.
Suppose you’re trying to build a recommendation engine for clothing, you’re training
it on one month of data (August), and you want to start generating recommendations
in the winter. One big issue is that the kinds of clothes people buy change from season to season: clothes buying is a nonstationary phenomenon over the scale of a few months. \\

Using machine learning trained on past data to predict the future is making the
assumption that the future will behave like the past. That often isn’t the case. 

\subsubsection{Choosing a measure of success}

Your metric for success will guide the choice of a loss function: what your model
will optimize. It should directly align with your higher-level goals, such as the success of your business. \\

For balanced-classification problems, where every class is equally likely, accuracy and area under the receiver operating characteristic curve (ROC AUC) are common metrics. For class-imbalanced problems, you can use precision and recall. \\

For
class-imbalanced problems, you can use precision and recall. For ranking problems or
multilabel classification, you can use mean average precision. And it isn’t uncommon
to have to define your own custom metric by which to measure success. \\

\subsubsection{Deciding on an evaluation protocol}

Once you know what you’re aiming for, you must establish how you’ll measure your
current progress.

\begin{itemize}
	\item \textit{Maintaining a hold-out validation set}--The way to go when you have plenty of data
	\item \textit{Doing K-fold cross-validation}--The right choice when you have too few samples for hold-out validation to be reliable
	\item \textit{Doing iterated K-fold validation}--For performing highly accurate model evaluation when little data is available
\end{itemize}

Just pick one of these. In most cases, the first will work well enough.


\subsubsection{Preparing your data}

here, we’ll assume a deep neural network:

\begin{itemize}
	\item your data should be formatted as tensors.
	\item The values taken by these tensors should usually be scaled to small values: for example, in the [-1, 1] range or [0, 1] range.
	\item If different features take values in different ranges (heterogeneous data), then the data should be normalized.
	\item You may want to do some feature engineering, especially for small-data problems.
\end{itemize}

Once your tensors of input data and target data are ready, you can begin to train models.

\subsubsection{Developing a model that does better than a baseline}

Your goal at this stage is to achieve \textit{statistical power}: that is, to develop a small model that is capable of beating a dumb baseline. In the MNIST digit-classification example, anything that achieves an accuracy greater than 0.1 can be said to have statistical power; in the IMDB example, it’s anything with an accuracy greater than 0.5. \\

Assuming that things go well, you need to make three key choices to build your
first working model:

\begin{itemize}
	\item \textit{Last-layer activation}--This establishes useful constraints on the network’s output. For instance, the IMDB classification example used \textit{sigmoid} in the last layer; the regression example didn’t use any last-layer activation; and so on.
	\item \textit{Loss function}--This should match the type of problem you’re trying to solve. For instance, the IMDB example used \textit{binary\_crossentropy}, the regression example used \textit{mse}, and so on.
	\item \textit{Optimization configuration}--What optimizer will you use? What will its learning rate be? In most cases, it’s safe to go with \textit{rmsprop} and its default learning rate.
\end{itemize}

[See table 4.1 at page 114]


\subsubsection{Scaling up: developing a model that overfits}

Once you’ve obtained a model that has statistical power, the question becomes, is your model sufficiently powerful? Does it have enough layers and parameters to properly model the problem at hand? \\

Remember that the universal tension in machine learning is between optimization and generalization; the ideal model is one that stands right at the border between underfitting and overfitting; between undercapacity and overcapacity. To figure out where this border lies, first you must cross it. \\

To figure out how big a model you’ll need, you must develop a model that overfits.
This is fairly easy:

\begin{enumerate}
	\item Add layers.
	\item Make the layers bigger.
	\item Train for more epochs.
\end{enumerate}

Always monitor the training loss and validation loss, as well as the training and validation values for any metrics you care about. When you see that the model’s performance on the validation data begins to degrade, you’ve achieved overfitting.
The next stage is to start regularizing and tuning the model, to get as close as possible to the ideal model that neither underfits nor overfits.


\subsubsection{Regularizing your model and tuning your hyperparameters}

This step will take the most time: you’ll repeatedly modify your model, train it, evaluate on your validation data (not the test data, at this point), modify it again, and repeat, until the model is as good as it can get. These are some things you should try:

\begin{itemize}
	\item Add dropout.
	\item Try different architectures: add or remove layers.
	\item Add L1 and/or L2 regularization.
	\item Try different hyperparameters (such as the number of units per layer or the learning rate of the optimizer) to find the optimal configuration.
	\item Optionally, iterate on feature engineering: add new features, or remove features that don’t seem to be informative.
\end{itemize}

Once you’ve developed a satisfactory model configuration, you can train your final
production model on all the available data (training and validation) and evaluate it
one last time on the test set. If it turns out that performance on the test set is significantly worse than the performance measured on the validation data, this may mean either that your validation procedure wasn’t reliable after all, or that you began overfitting to the validation data while tuning the parameters of the model. In this case, you may want to switch to a more reliable evaluation protocol (such as iterated K-fold validation).

















































































































































\end{document}